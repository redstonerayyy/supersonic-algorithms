\documentclass[12pt,a4paper]{article}

% packages, images, math
\usepackage{geometry, graphicx, amsmath, amsfonts, array}

% colors
\usepackage[dvipsnames]{xcolor}

% for urls
\usepackage[colorlinks=true,urlcolor=ProcessBlue,linkcolor=ForestGreen]{hyperref}

% Remove Indentation at new line
\setlength{\parindent}{0cm}

% Set Font to Arial
% \usepackage{fontspec}
% \setmainfont{Arial}

% Set Font to Helvet
\usepackage{helvet}
\renewcommand{\familydefault}{\sfdefault}

% Set Layout
\geometry{
    a4paper,
    left=25mm,
    right=25mm,
    top=25mm,
    bottom=20mm
}

% Redefine title
\makeatletter
\def\@maketitle{%
  \newpage
  \null
  \vskip 2em%
  \begin{center}%
  \let \footnote \thanks
    {\Huge\bfseries\@title \par}%
    \vskip 1.5em%
    {\large
      \lineskip .5em%
      \begin{tabular}[t]{c}%
        \@author
      \end{tabular}\par}%
    \vskip 1em%
    {\large \@date}%
  \end{center}%
  \par
  \vskip 1.5em}
\makeatother

\begin{document}

\title{Supersonic Algorithms}
\author{Anton Rodenwald}

\maketitle

\clearpage
\section*{Kurzfassung}
Nachdem wir im Informatikunterricht der SEK II Sortieralgorithmen behandelt hatten,
stellte ich mir die Frage, wie man am schnellsten eine Liste von 10 Millionen zufällig Generierten Zahlen
sortieren kann und welche Programmiersprache und welche Techniken man nutzen sollte.
Daraus entwickelte sich dann die etwas allgemeinere Fragestellung, nämlich welche Optimierungen
erhöhen die Ausführgeschwindigkeit von Programmen am meisten und wieso?
Mir war bekannt, dass Python, was wir im Unterricht verwendet hatten, als eine der langsamsten Sprachen gilt, 
weswegen ich neben Python auch noch C++ wählte, was allgemein als eine der schnellsten Sprache gilt.
Ich implementierte anschließend verschiedene Variationen der Quicksort und anderer Algorithmen und testete so, 
in welchem Maß Optimierungsansätze die Performance beeinflussten.
Dabei kam ich zu dem Ergebnis, dass die besten Python Bibliotheken zur Optimierung "numpy" und "numba" waren,
wobei C++ trotzdem schneller war, womit sich meine Hypothese bestätigte.
Dies erklärte ich mir dadurch, dass die Python eine Interpretierte und C++ eine kompilierte Sprache ist, diese
beiden also gänzlich verschiedenen und somit auch die Möglichkeiten zur Optimierung total verschieden sind.
Schlussendlich gelang es mir noch unter Nutzung von AVX2, in C++ eine 4x schnellere Version als die standardmäßig 
Vorhande zu entwickeln, einem speziellen Befehlssatz, was mir zeigte, dass es im Gebiet der Codeoptimierung
noch viel zu entdecken gibt.

\clearpage
\section*{Inhaltsverzeichnis}

\hyperref[sec:einleitung]{Einleitung}

\hyperref[sec:methode]{Vorgehensweise, Materialien, Methode}

\clearpage
\section*{1. Einleitung}
\label{sec:einleitung}

Im Informatik Leistungskurs des 12 Jahrgang beschäftigten wir uns nach den Herbstferien mit
der Laufzeit von Algorithmen am Beispiel der Quicksort, einem Sortieralgorithmus. 
Nach diesem thematischen Impuls ergab sich mein Projekt zur Erforschung von Sortieralgorithmen, 
wobei ich mich schnell auf die Aspekte der Implementation und tatsächlichen Ausführung 
von Algorithmen fokussierte, da Sortieralgorithmen algorithmisch bereits sehr weit erforscht sind.
Ich entschied mich deswegen, nicht nach besseren Algorithmen zu suchen, sondern nach Wegen, 
mein ursprüngliches Program in Python in seiner Ausführung zu beschleunigen und so vorteilhafte
Wege der Geschwindigkeitsoptimierung zu entdecken.
Ich stellte mir die Frage, welche Optimierungen die Ausführgeschwindigkeit von Programmen
am meisten erhöhen und wieso?
Außerdem wollte ich ein möglichst schnelles Program zur Sortierung implementieren, weswegen ich
neben Python, einer interpretierten und als langsam geltenden Sprache, noch C++, 
eine hochperformante, kompilierte Sprache wählte.
Zusätzlich entschied ich mich noch, in einigen anderen Sprachen meine Quicksort zu implementieren,
weil mich ein Vergleich verschiedner Sprachen interessierte.

\clearpage
\section*{Vorgehensweise, Materialien, Methode}
\label{sec:methode}

testen

zeitmessung


software versionen

python, c++, java, javascript, lua, go, julia



\clearpage
\section*{Ergebnisse}
\clearpage
\section*{Diskussion}
\clearpage
\section*{Zusammenfassung}
\clearpage
\section*{Literaturverzeichnis}

\begin{thebibliography}{10}
    \bibitem{terdiman2000}
        Pierre Terdiman,
        \textit{Radix Sort Revisited},
        \url{http://codercorner.com/RadixSortRevisited.htm}
        \textit{Zugriff am: 10.1.23},
        \textit{Veröffentlicht am 4.01.2000}
    
    \bibitem{michael2001}
        Michael Herf,
        \textit{Radix Tricks},
        \url{http://stereopsis.com/radix.html}
        \textit{Zugriff am: 10.1.23},
        \textit{Veröffentlicht im December 2001}

\end{thebibliography}

\end{document}