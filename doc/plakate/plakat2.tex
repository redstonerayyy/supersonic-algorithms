\documentclass[25pt, a0paper, portrait]{tikzposter}

% package
\usepackage{graphicx}
\usepackage{fancyhdr}
\usepackage{xcolor}
\usepackage{bchart}
\usepackage{tikz}
\usepackage{multicol}
\usepackage{enumitem}
\usepackage{setspace}

% code highlighting
\usepackage{listings}
\usepackage{color}

\definecolor{dkgreen}{rgb}{0,0.6,0}
\definecolor{gray}{rgb}{0.5,0.5,0.5}
\definecolor{mauve}{rgb}{0.58,0,0.82}

\lstset{frame=tb,
  language=Python,
  aboveskip=3mm,
  belowskip=3mm,
  showstringspaces=false,
  columns=flexible,
  basicstyle={\normalsize\ttfamily},
  numbers=none,
  numberstyle=\tiny\color{gray},
  keywordstyle=\color{blue},
  commentstyle=\color{dkgreen},
  stringstyle=\color{mauve},
  breaklines=true,
  breakatwhitespace=true,
  tabsize=3
}

\usepackage{filecontents}
\begin{filecontents*}{foo.py}
import time

class Timer:
    timers = {}
    @staticmethod
    def start(name):
        Timer.timers[name] = time.time_ns()
    
    @staticmethod
    def stop(name):
        diff = time.time_ns() - Timer.timers[name]
        diffseconds = diff / (10 ** 9) 
        print(name, ":", round(diffseconds, 5), "Seconds")
\end{filecontents*}

% language
\usepackage[german]{babel}
\usetikzlibrary{babel}

% font to arial
\usepackage{unicode-math}
\usepackage{fontspec}
\setmainfont{Arial}

% custom theming
% https://github.com/debjyoti385/tikzposter

% define colorstyle
\definecolorstyle{Rodenland}{
    % Define default colors
    % PurpleGrayBlue
    \definecolor{colorOne}{HTML}{AE0D45}
    \definecolor{colorTwo}{HTML}{7F8897}
    \definecolor{colorThree}{HTML}{C8512D}
}{
     % Background Colors
    \colorlet{backgroundcolor}{white}
    \colorlet{framecolor}{white}
    % Title Colors
    \colorlet{titlebgcolor}{colorOne}
    \colorlet{titlefgcolor}{white}
    % Block Colors
    \colorlet{blocktitlebgcolor}{colorTwo}
    \colorlet{blocktitlefgcolor}{colorOne}
    \colorlet{blockbodybgcolor}{white}
    \colorlet{blockbodyfgcolor}{black}
    % Innerblock Colors
    \colorlet{innerblocktitlebgcolor}{colorThree}
    \colorlet{innerblocktitlefgcolor}{white}
    \colorlet{innerblockbodybgcolor}{white}
    \colorlet{innerblockbodyfgcolor}{black}
    % Note colors
    \colorlet{notefgcolor}{black}
    \colorlet{notebgcolor}{colorOne!20!white}
    \colorlet{notefrcolor}{colorOne!00!white}
 }

 \definetitlestyle{Default}{
    width=500mm, roundedcorners=30, linewidth=0.4cm, innersep=1cm,
    titletotopverticalspace=15mm, titletoblockverticalspace=20mm,
    titlegraphictotitledistance=10pt, titletextscale=1
}{
    \begin{scope}[line width=\titlelinewidth, rounded corners=\titleroundedcorners]
        \draw[color=framecolor, fill=titlebgcolor]%
        (\titleposleft,\titleposbottom) rectangle (\titleposright,\titlepostop);
    \end{scope}
}

 % define block stzle
 \defineblockstyle{Minimal}{
    titlewidthscale=1, bodywidthscale=1, titleleft,
    titleoffsetx=0pt, titleoffsety=0pt, bodyoffsetx=0pt, bodyoffsety=0pt,
    bodyverticalshift=0pt, roundedcorners=0, linewidth=0.2cm,
    titleinnersep=1cm, bodyinnersep=1cm
}{
    \begin{scope}[line width=\blocklinewidth, rounded corners=\blockroundedcorners]
       \ifBlockHasTitle %
           \draw[draw=none]%, fill=blockbodybgcolor]
               (blockbody.south west) rectangle (blocktitle.north east);
        %    \draw[color=blocktitlebgcolor, loosely dashed]
        %        (blocktitle.south west) -- (blocktitle.south east);%
       \else
             \draw[draw=none]%, fill=blockbodybgcolor]
                 (blockbody.south west) rectangle (blockbody.north east);
        \fi
    \end{scope}
}

% inner block style
\defineinnerblockstyle{Default}{
    titlewidthscale=1, bodywidthscale=1, titlecenter,
    titleoffsetx=0pt, titleoffsety=0pt, bodyoffsetx=0pt, bodyoffsety=0pt,
    bodyverticalshift=0pt, roundedcorners=20, linewidth=4pt,
    titleinnersep=12pt, bodyinnersep=12pt
}{
    \begin{scope}[line width=\innerblocklinewidth, rounded
      corners=\innerblockroundedcorners, solid]
        \ifInnerblockHasTitle %
           \draw[color=innerblocktitlebgcolor, fill=innerblocktitlebgcolor]
           (innerblockbody.south west) rectangle (innerblocktitle.north east);
           \draw[color=innerblocktitlebgcolor, fill=innerblockbodybgcolor]
           (innerblockbody.south west) rectangle (innerblockbody.north east);
        \else
           \draw[color=innerblocktitlebgcolor, fill=innerblockbodybgcolor]
           (innerblockbody.south west) rectangle (innerblockbody.north east);
        \fi
    \end{scope}
}

 % define layout theme
\definelayouttheme{PosterAnton}{
    \usecolorstyle{Rodenland}
    \usebackgroundstyle{Default}
    \usetitlestyle{Default}
    \useblockstyle{Minimal}
    \useinnerblockstyle{Default}
    \usenotestyle{Default}
}

% Theme Simple
\usetheme{PosterAnton}

% change background color of poster
\colorlet{backgroundcolor}{white}

% remove tikzposter notice at bottom
\tikzposterlatexaffectionproofoff

\settitle{ \centering \vbox{
    \@titlegraphic \\[\TP@titlegraphictotitledistance] \centering
    \color{titlefgcolor} {\bfseries \Huge \sc \textbf{\@title} \par}
    \vspace*{1em}
    {\huge \textbf{\@author} \par} \vspace*{1em} {\LARGE \@institute}
}}

% formatting
\setlength\columnsep{3cm}
\setstretch{1.25}

% redefine title
\makeatletter
\newcommand\insertlogoi[2][]{\def\@insertlogoi{\includegraphics[#1]{#2}}}
\newcommand\insertlogoii[2][]{\def\@insertlogoii{\includegraphics[#1]{#2}}}
\newlength\LogoSep
\setlength\LogoSep{0pt}

\insertlogoi[width=5cm]{example-image-a}
\insertlogoii[width=5cm]{example-image-b}

\renewcommand\maketitle[1][]{  % #1 keys
    \normalsize
    \setkeys{title}{#1}
    % Title dummy to get title height
    \node[transparent,inner sep=\TP@titleinnersep, line width=\TP@titlelinewidth, anchor=north, minimum width=\TP@visibletextwidth-2\TP@titleinnersep]
        (TP@title) at ($(0, 0.5\textheight-\TP@titletotopverticalspace)$) {\parbox{\TP@titlewidth-2\TP@titleinnersep}{\TP@maketitle}};
    \draw let \p1 = ($(TP@title.north)-(TP@title.south)$) in node {
        \setlength{\TP@titleheight}{\y1}
        \setlength{\titleheight}{\y1}
        \global\TP@titleheight=\TP@titleheight
        \global\titleheight=\titleheight
    };

    % Compute title position
    \setlength{\titleposleft}{-0.5\titlewidth}
    \setlength{\titleposright}{\titleposleft+\titlewidth}
    \setlength{\titlepostop}{0.5\textheight-\TP@titletotopverticalspace}
    \setlength{\titleposbottom}{\titlepostop-\titleheight}

    % Title style (background)
    \TP@titlestyle

    % Title node
    \node[inner sep=\TP@titleinnersep, line width=\TP@titlelinewidth, anchor=north, minimum width=\TP@visibletextwidth-2\TP@titleinnersep]
        at (0,0.5\textheight-\TP@titletotopverticalspace)
        (title)
        {\parbox{\TP@titlewidth-2\TP@titleinnersep}{\TP@maketitle}};

    % \node[inner sep=0pt,anchor=west] 
    %   at ([xshift=-\LogoSep]title.west)
    %   {\@insertlogoi};

    \node[inner sep=0pt,anchor=east, right=-8cm] 
      at ([xshift=\LogoSep]title.east)
      {\@insertlogoii};

    % Settings for blocks
    \normalsize
    \setlength{\TP@blocktop}{\titleposbottom-\TP@titletoblockverticalspace}
}
\makeatother

% set mono font
% \setmonofont{Consolas}

% main document
\begin{document}

% title with logo
\insertlogoii[width=5cm]{apple.jpg}

\title{Supersonic Algorithms}
\author{Anton Rodenwald (18), Schillerschule Hannover}

\maketitle

\begin{columns}
    \column{0.5}
    \block{Ideenfindung und Forschungsfrage}{
        \Large
        \begin{itemize}
            \item Sortieralgorithmen für Zahlen im Informatikunterricht kennengelernt
            \item Ich fragte mich, wie man Zahlen am schnellsten sortieren kann
            \item Sortieralgorithmen sind bereits bekannt, deswegen Fokus auf Implementation
            \item Forschungsfrage: Wie lassen sich Programme durch geschickte Implementation
                  in ihrer Ausführung beschleunigen?
            \item Kein Vergleich von Sortieralgorithmen, fast immer Quicksort verwendet
            \item Fokus auf die Programmiersprachen Python, C++ und C
            \item In Interquellen wurde C++ um ein vielfaches schneller als Python beschrieben
            \item Es war auch meine Vermutung, dass C++ deutlich schneller Zahlen sortiert
        \end{itemize}
    }

    \column{0.5}
    \block{Material, Vorgehen und Methode}{
        \Large
        \begin{itemize}
            \item Ich nutze meinen PC und Laptop zur Implementation
            \item Ich verwende die Programme Visual Studio Code und Compiler oder Interpreter für meine Programmiersprachen
            \item Ich implementierte anschließend die Quicksort auf verschiedene Arten und erhielt so viele
                  Variationen zum testen
            \item Die Tests führte ich auf meinem Desktop PC durch (CPU ist der Ryzen 7 2700, 16 GB Arbeitsspeicher), der immer gleich ausgelastet war,
                  um Vergleichbarkeit zu gewährleisten
            \item Die Zeit stoppte ich mit den Zeitfunktionen der Sprachen
        \end{itemize}
        \vspace{1cm}
        \textbf{Beispiel der Zeitnahme in Python}
        \begin{tikzfigure}
            \lstinputlisting{foo.py}
        \end{tikzfigure}
    }
\end{columns}
\begin{columns}
    \column{0.5}
    \block{Ergebnissdiskussion}{

    }

    \column{0.5}
    \block{Schwierigkeiten}{
        \Large
        \begin{itemize}
            \item Meine größte Schwierigkeit war, dass ich noch nicht viel Erfahrung
                  mit der Optimierung und dem wissenschaftlichen Arbeiten hatte
            \item Erstellung der Dokumentation mit \LaTeX teils frustrierend
            \item Mit einigen Bibliotheken und Sprachen schwierigkeiten
            \item Mir war das testen mit 10 Millionen Zahlen in Go nicht möglich,
                  da dort Rekursion nicht so gut funktioniert
        \end{itemize}
    }
    \block{Beantwortung der Forschungsfrage und Ausblick}{
        \Large
        \begin{itemize}
            \item Die besten Möglichkeiten zur Optimierung in Python sind NumPy und Numba
            \item Cython schlechter, CTypes komplex
            \item In C++ bietet AVX2 interessante Möglichkeiten der Optimierung
            \item Python kann ähnlich schnell wie C++ sein kann
            \item Insgesamt profitiert Python vorallem von schnellen C
                  Bibliotheken
        \end{itemize}
        \includegraphics{memepy.jpg}
    }
\end{columns}


\end{document}
